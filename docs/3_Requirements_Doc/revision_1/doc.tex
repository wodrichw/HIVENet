\documentclass[onecolumn, draftclsnofoot,10pt, compsoc]{IEEEtran}
\usepackage{graphicx}
\usepackage{url}
\usepackage{setspace}
\usepackage{geometry}
\geometry{margin = 0.75in}

% ---- Removes "References" from bibliography section ----
\usepackage{etoolbox}
\patchcmd{\thebibliography}{\section*{\refname}}{}{}{}

\title{Requirements Document for HIVENet}
\author{Enrique Ferndandez, Alex Garcia, William Wodrich, Andrew Davis\\CS 461 Oregon State University\\Fall 2018}

% ---- New sentences on new lines because YOLO. So that's why the format may look weird ----

\begin{document}
\pagestyle{plain}

\begin{titlepage}
	\maketitle

	\pagenumbering{arabic}
	\pagestyle{plain}

%---- Begin Abstract ----
	\noindent
	\textbf{Abstract} \\
	\indent
Type here

\end{titlepage}

\newpage
\tableofcontents
\newpage

% ---- Section I ----
\section{Introduction}
This software requirement specification describes the scope of our project and an overview of the requirements this project will meet.
A list of abbreviations and definitions is provided within this section.

	\subsection{Purpose}
The purpose of this document is to give a detailed description of the requirements for the Hierarchical Information Variant Exchanging Network (HIVENet).
This document will explain the purpose and development of the system as well as provide a high-level understanding of HIVENet's infrastructure, restraints, and interface.

	\subsection{Scope}
The HIVENet facial recognition software application is designed to demonstrate a new method to optimize neural network training and illustrate methods to enable communication between neural networks.
The tool will be built to interface with Google's Deep Learning API, TensorFlow\texttrademark .

	\subsection{Glossary}
		\begin{itemize}
			\item \textbf{HIVENet: }
"Hierarchical Information Variant Exchanging Network." 
Used throughout the document to refer to the project application.

			\item \textbf{Edge-Device: }
A computer wirelessly connected to the core-device.
Equipped with a camera to receive video input.

			\item \textbf{Edge-Process: }
The tandem structure of the artificial neural network and inference engine housed on an edge device.

			\item \textbf{Core-Device: }
A central computer wirelessly connected to all edge-devices.

			\item \textbf{Core-Process: }
The HIVENet application manager, housed on the core-device.
It periodically receives updates from edge-proccesses and handles the variant exchange process.

			\item \textbf{Neural Network: }
"NN."
A computational framework that employs several learning algorithms to process complex data input.
This framework is generally implemented as directed weighted graphs of artificial neurons.

			\item \textbf{Artificial Neurons: } 
The most basic component of a neural network.
They mirror neurons in the brain, acting as aids in the overall decision process of NNs \cite{glossary_one}.

			\item \textbf{Coefficients/Weights: }
The values inside of a NN that define and modifiy its behavior.
They are associated with the artificial neurons and edges that connect them.

			\item \textbf{Training: }
The act of providing a NN with a dataset to operate on.

			\item \textbf{Inference Engine: }
TBD.

			\item \textbf{Fitness Level: }	
An enumeration of an edge-process' competency.
As the NNs train themselves on a dataset, it will modify itself to increase accuracy, raising the fitness level.
		\end{itemize}

	\subsection{References}
% ---- Bibliography ----
\begin{thebibliography}{10}
	
	\bibitem{glossary_one}
	Eyal Reingold,
	"Artificial Neural Networks Technology,"
	\textit{University of Toronto Mississauga},
	[Online].
	Available: \\\textit{http://www.psych.utoronto.ca/users/reingold/courses/ai/cache/neural\_ToC.html}.
	[Accessed October 31 2018]

\end{thebibliography}
% --- End Bibliography ----

	\subsection{Overview}
The following sections will provide an overview of HIVENet's infrastructure and system functionality. 
Chapter two will also mention user interaction and assumptions we have associated with our client.
Finally, any constraints measure on the application will be covered.

	\newpage
% ---- End Section I ----


% ---- Section II -----
\section{Overall Description}
This section will give a high-level HIVENet software infrastructure and front-end usability.
This software will be explained in a matter of how the HIVENet infrastructure will be setup, how users will interact with HIVENet, and what data will be passed.
By the end, assumptions and constraints for the system will be presented.

	\subsection{Product Perspective}
This application will consist primarily of underlying code of hierarchical NN stack that will pass datasets, coefficients, and alter edge-devices inference engines that will allow for those same edge-devices to be able to identify the users that are using this produce across the entire network.
The user will interact with an authentication GUI webpage. % --- Discuss ---

	\subsection{Product Functions}
HIVENet is designed to register and recognize any user who interacts with this application. 
It uses a hierarchical, convolutional NN that self-trains its edge-devices.
Each edge-device will have an established communication channel to the centeral-device.
This channel will be the primary communication to share NN coefficients and datasets to all other edge-devices.
The central-device's functionality will be to take all end-devices and choose the one with the greatest fitness level.
That node will then reconfigure the inference engines on edge-devices to further optimize the success rate of the facial recognition software.

	\subsection{User Characteristics}
With HIVENet, the user will be able to register with an edge-device and create an account that will contain their first and last name.
Users can register with physical facial characteristics as their form of identification.
That same user will then be able to login into any other edge-devices.

	\subsection{Assumptions and Dependencies}
Our facial recognition software will initially distinguish three facial features: eyebrows, eye color, and lips.

	\subsection{Constraints}

	\subsection{Apportioning of Requirements}

	\newpage
% ---- End Section II ----


% ---- Section III ----
\section{Specific Requirements}
	\subsection{Interfaces}
		\subsubsection{User Interfaces}
		\subsubsection{Software Interfaces}
	\subsection{Functional Requirements}
This section includes the requirements that specify the fundamental actions of the software system, and the necessary outcomes that are expected by the client.


\section{Requirements Overview Notes: (NOT IN COMPLETED DRAFT)}
\subsection{Ben Had}
\begin{itemize}
		\item functional requirements
		\item performance requirements
	
\end{itemize}
\subsection {HIVENet cares about both of these:}
	\subsubsection{functional requirements}
	\begin{itemize}
		\item all edge devices need to have two way communication with core device
		\item All edge devices need to be able to train neural networks
		begin\begin{itemize}
			\item Image processing
			\item Data filtering and prepping (multiple manipulated images from one image)
		\end{itemize}
		\item Core device needs to be able to evaluate which features on each edge device would advantage all other devices if merged into the core neural net
		\item Core device needs a method of merging a feature of a edge process into the core process
		\item Edge device needs hold an interface which allows for core device to evaluated it's fitness (interface will expose values of certain features)
		\item A process on the core device needs to have a method which determines when each edge device will be evaluated
		\item Core device needs a way to package updated neural net and send package to edge devices
		\item Edge devices need to be able to incorporate new package from core device into neural net process.		
		\item Edge devices need to be uniquely identafiable
		\item A meathod to associate image attributes with each image needs to be in place
		\begin{itemize}
			\item these would be like eye color, eye shape etc
			\item could have user accounts
			\begin{itemize}
				\item storing user account info (login name, and account details)
				\item login screen
				\item form for entering user info (eye color, eyebrow color, eye shape)
			\end{itemize}
		\end{itemize}
		\item Once the inference engine has guessed eye color, eye shape, and eyebrow color, program can return a filtered list of all users that have that eye color, eye shape, and eyebrow color
		\item core device must have the most up-to-date inference engine
	\end{itemize}

	\subsubsection{Performance Requirements}
	\begin{itemize}
		\item core process can identify eye color with 60\% success rate -goal 80%
		\item core process can identify shape color with 60\% success rate -goal 80%
		\item core process can identify eyebrow color with 60\% success rate -goal 80%
		\item User can authenticate in under 30 seconds, goal 15 seconds
		\item User system can register a user in under a minute goal 15 seconds
		\item Neural networks should be able to train in under 24 training hours, goal 8 hours
		\item Integration of features from the localized processes into the core NN should improve the accuracy of the core NN 80\% of the time, goal 95\% of the time
		\item The size of the feature update package sent from a localized process to the core NN should be 20\% of the data required to achieve that feature
	\end{itemize}
\textbf{ID: FR1} \\
	TITLE: \\
	DESC: \\
	DEPEND: \\

	\newpage
% ---- End Section III ----



\end{document}